\pagebreak
\section{Introduction}
\label{sec:introduction}


For this laboratory assignment, our objective was to study a circuit containing 11 components. These include seven resistors, one capacitor, one current source (voltage-controlled), a current-controlled voltage source and an independent sinusoidal voltage source. This independent source has a behavior based on the following equation: 
\begin {equation}
V_s(t) = V_s u(-t) + sin(2\pi f t)u(t)
\end {equation}
We studied the circuit, shown in Figure 1~\ref{fig:circuit}, using a structured approach, based on the nodal method, as explained in the theory classes.

To analyze and study the given information we first proceeded to identify the eight nodes present in the circuit. Furthermore, we extracted from the Python script the values for the resistors and the capacitor, as well as the values of the constants related with the independent source and the initial value (linked to t<0). 

Through the report, in section~\ref{sec:analysis}, we present a theoretical analysis of the circuit, which is divided in six different subsections, each of them corresponding to a different step given in the guide to solve this problem. In addition, in section~\ref{sec:simulation}, we run a simulation of the circuit, using NGSpice, whose results are then compared to the theoretical results obtained in Section~\ref{sec:analysis}. Finally, the conclusions of this study are presented in Section~\ref{sec:conclusion}.

\begin{figure}[h] 
	\centering
	\includegraphics[width=0.8\linewidth]{t2}
	\caption{Circuit that will be analyzed in this lab.}
	\label{fig:circuit}
\end{figure}

