\section{Conclusion}
\label{sec:conclusion}

In this laboratory assignment, the objective was to analyze the circuit in Figure~\ref{fig:circuit}, using two different structured methods: the Nodal and the Mesh Methods. 

The calculations were solved theoretically by using Octave, and the circuit simulation was done using the NGSpice tool.

As stated in Section~\ref{sec:simulation}, the simulation results matched the theoretical results
precisely, as we expected, given that the studied circuit only contained linear components; therefore, the theoretical
and simulation models shouldn't differ. 

However, we cannot say that the models we used are correct and represent the reality. In fact, if we had worked with real components (instead of ideal ones, as assumed by the used model), we would be most likely to obtain results different from the theoretical ones, as each component, and even the measurement equipment and the wiring cables, is related to uncertainties and passible to origin errors that would contaminate the extracted values. All we can conclude from this lab is that the two theoretical methods are in alignment because they gave us the same results, and we can also state that NGSpice probably does its calculations following the same theoretical background as the Nodal and the Mesh Methods, given that the results generated by the program were equal to the theoretical ones.
This laboratory assignment was also important from the perspective that it was our first contact with some tools of invaluable utility for any engineer, such as GitHub, LaTeX and, of course, NGSpice.
